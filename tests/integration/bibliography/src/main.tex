\documentclass{article}

\usepackage[prefix=BN]{basicnotes}

\BNsetTitle{Test documents}

\BNtitle{A test document for the simplebibliography package}

\BNauthor{N. Raghavendra}

\BNdate{2020-08-21}

\usepackage[prefix=SB]{simplebibliography}

\SBaddbibresource{main.bib}

\begin{document}

\BNmakeTitle

\begin{abstract}
  This document is a test for the simplebibliography package.
\end{abstract}

\tableofcontents

\section{Hello world}
\label{sec:hello-world}


The phrase \SBcite{borovik:2010:mathem_under_the_micros} is an example
of a bibliographic citation.  It cites a book with one author.

The article \SBcite{bourbaki:univer} is a part of a book.  The book
must appear in the list of references even though it is not cited in
this document.

The text \SBcite{dwyer:2004:homot_limit_funct_on_model} is a book with
four authors.

\section{Hello universe}
\label{sec:hello-universe}

The citation \SBcite{altenkirch:2002:α_conver_is_easy} is for an
online unpublished article.

The article \SBcite{farmer:2008:seven_virtues_of_simpl_type_theor} was
published in a journal.

The citation \SBcite[Section~3, page~55]{bourbaki:univer} includes a
note about the target location in the cited reference.

\SBprintbibliography

\end{document}
